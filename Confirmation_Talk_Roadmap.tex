\documentclass[12pt]{report}
\usepackage{graphicx}
\usepackage{amsfonts}
\usepackage{float}
\usepackage{pifont}
\usepackage{hyperref}
\usepackage{pdfpages}
\usepackage{caption}
\usepackage{relsize}
\usepackage{amssymb}
\usepackage{multicol}
\usepackage{bigints}
\usepackage{mathtools}
\usepackage{subcaption}
\usepackage[utf8]{inputenc}
\usepackage{flexisym}
\usepackage{cite}
\usepackage{siunitx}   %For degree
\usepackage{IEEEtrantools}
\usepackage{amsmath}      % for \tag and \eqref macros\\
\usepackage{tgschola}
\usepackage{ amssymb }
\usepackage{empheq}  %boxed over two equ.
\usepackage[T1]{fontenc}
\usepackage[printwatermark]{xwatermark}
\usepackage[margin=0.9in]{geometry} %margin-------------margin
%\newwatermark[allpages,color=red!50,angle=45,scale=1,xpos=0,ypos=0]{The Cosmological Cafe}
\newcommand\vertarrowbox[3][6ex]{%
  \begin{array}[t]{@{}c@{}} #2 \\
  \left\uparrow\vcenter{\hrule height #1}\right.\kern-\nulldelimiterspace\\
  \makebox[0pt]{\scriptsize#3}
  \end{array}%
}
\begin{document}
%\tableofcontents 
%\newpage
\begin{Large}
\textbf{Confirmation Talk Roadmap-October 27th, 2020} 
\end{Large}
\vspace{5mm}
\begin{itemize}
 \item[\ding{74} {\footnotesize Slide-0a}]
 Good afternoon to all of you. I Darshan, am a PhD student, registered  under the supervision of Dr Deepak Jain and Prof Shobhit Mahajan.  Today, here in this confirmation talk, I'm going to discuss some applications of Strong gravitational lensing in cosmology.
 \item[\ding{74} Slide-1]
 First I'll discuss some basic cosmology and distances in cosmology. Then will introduce strong gravitational lensing effect and finally end up with results and conclusions. 

\item[\ding{74} Slide-2]
In science, the structure of any theory depends on some fundamental principles or assumptions. Similarly, cosmology is based on the fundamental assumptions or principle of simplicity, namely, cosmological principle. On the largest scales, the universe is assumed to be uniform. This idea is called the cosmological principle. The cosmological principle state that at large scale i.e scale greater than 100MPc, universe is spatially homogeneous and isotropic. By homogenous, I mean that when we look at different locations, universe appears to be same and by Isotropic, I mean that it is same in all directions. The assumptions of large scale homogeneity and isotropy appears to be an accurate description of our universe. Under this assumption, Friedmann-Lema\^{i}tre-Robertson-Walker gave a metric that is called FLRW metric. This FLRW metric describes the geometry of universe as shown here. Where, $a(t)$ is scale factor of dimension length, which describe how distances in a homogeneous and isotropic universe expand or contract with time and $t, r, \theta, \phi$ are dimensionless coordinates defined as comoving coordinates. To relate the geometry of the universe to its mass energy content, Einstein gave an equation known as Einstein’s equations. The First two term of left hand side tells about curvature and can be calculated from the Riemannian tensor. $\Lambda$ in this equation is called cosmological constant. Right hand side of this equation represents energy momentum stress tensor. Basically, These equations tells us how space-time will behave in the presence of energy-momentum \textit{or in the words of John Wheeler: ”Mass tells space-time how to curve and space-time tells mass how to move”.} To describe the matter part, we consider a perfect fluid. This kind of fluid just is convenient becuase perfect fluids are isotropic in their rest frame, which satisfy cosmological principle and for a perfect fluid, this energy momentum stress tensor is given by this equation, where P and $\rho$ are pressure and density of perfect fluid in their rest frame.

\item[\ding{74} Slide-3]
So we defined the main three ingredients of cosmology, which are FLRW metric, Einstein Equations and Energy-momentum-stress tensor. Using these three quantity Friedmann derive two equations, which is known as Friedmann's Equations. First equation can be calculated from 00 component of Einstein’s field equations and second by  $ij$th component of Einstein equation. Basically, Friedmann equations describe the evolution of the scale factor with time.  If we club these two equations, we can write single equation as shown here. Left hand of this equation denotes the acceleration. The first term of right hand side is pressure and density term or gravity term. The negative sign denotes that gravity slow down the expansion.\\
\textit{Now lets discuss about the $\Lambda$ term- In order to justify the Einstein's idea of a static universe, there should be a term to compensate the gravity or we mean that we have to put an antigravity term so that the universe become static otherwise it does collapse. But in 1929, Hubble declare that universe is expanding rather than a static. So this led Einstein to gave up the idea of the cosmological constant and to reproduce an expanding universe model which was consistent with the observations to that date. Further in 1990s, the CMB observations (of temperature fluctuations and $H_0$ values) carried by COBE and HST began to suggest that energy density of universe is dominant by cosmological constant and a new model came into the picture-$\Lambda$CDM model. And also between 1998 and 1999 supernova project provided a strong evidence of an accelerated expansion. All these factors make the reapperance of the Einstein's cosmological constant.
where cosmological constant leads to accelerate the universe.} 

\item[\ding{74} Slide-4]
Using Friedmann Equation one can easily define the Hubble parameter which is ratio of $\dot{a}/a$. which describes the present-day expansion rate of the Universe. Hubble parameter is vary with redshift as shown here. Where z is redshift. Redshift is a very important phenomenon in cosmology and astronomy. This phenomenon occurs when an object emit or reflect the electromagnetic radiation is shifted towards the less energetic (higher wavelength) end of spectrum. 
%This is called cosmological redshift and is represented by z. 
The observation of shifts in frequency of light emitted by distant sources gives the most important information about the redshift. And finally from redshift we can easily define the cosmic scale factor a(t) as shown here. To move into next section, let me define some cosmological density parameters that are radiation, matter, curvature and dark energy density parameters. And sum of all the density parameters is equal to one.

\item[\ding{74} Slide-5]
The most basic and difficult measurement in cosmology is distance. So the basic aspects is that all distances somehow measure the separation between events. In a curved and expending universe, the measurement of distance is a tricky business.  Becuase in the expending universe, the distances between objects are constantly changing.  Therefore, distances measurement have played a key and surprising role for a dynamical universe.


In Cosmology, there are many ways to specify the distances between two points. First we define the comoving distance: Lets say two points are moving with Hubble flow, their separation distance between them remains constant in all epoch and this separation distance is called comoving distance. Alternatively, it is the distance between them which is measured with rulers divided by the ratio of the scale factor of the universe then to now. Mathematical, it can be written as shown here. which is not directly observable.


To measure distance from a distant object, we require some standard population of object i.e .objects of same or known size and objects of same luminosity. For a known size of objects we can use them as standards ruler and correspond to standard ruler we can define angular diameter distance as shown here. On the other hand, for a same luminosity objects we can use them as standards candle. And correspond to standard candle we can define luminosity distance as shown here. 

\item[\ding{74} Slide-6]
These all three distances are related to each other via redhsift as shown here which also depends on the different combinations of cosmological parameters. And also we can see the behaviour of each distance with redshift.

\item[\ding{74} Slide-7]
Suppose a source is located far away from us and we are here on earth are observers. Further, we can imagine that there are many massive objects are present between source and observer. So when a light emitted from source and coming towards observer, then due to presence of mass object or gravity, light gets bent.
%{When the light rays from source is passes near a mass distribution, they get bent due to the presence of gravity.} 
Light from a source gets deflected towards observer and we see multiple images of a source. The effect of multiple images because of deflection of light rays is called strong gravitational lensing and the intervening mass or matter acts as a lens, as shown in figure with observer point as Chandra and a view of multiple images are shown on right hand side over here. 
%In a special case all three observer, lens and source are on the same line then we get a Ring like image that is called Einstein Ring with radius called Einstein radius.

 \item[\ding{74} Slide-8]
In the above figure, the deflection of light is depends only on the mass distribution of the galaxy that is acting as a lens in the lensing systems. Therefore, to understand the lensing effect, it is crucial to study the mass distribution in lens galaxy. In this work, we're going to discuss the different type of lens profiles of a galaxy. Three lens profiles are listed which will study one by one.

\item[\ding{74} Slide-9] 
The geometry of strong lensing systems is shown here. Where observer, lens and source is denoted by the label O, L and S with their angular diameter distances as $d_{A}^{^{ol}}$, $d_{A}^{^{ls}}$ and $d_{A}^{^{os}}$. In this, we assumed a point like lens of a lensing systems. When the light is comes from a position S then before reaching to observer, light gets bent because of potential of the lens at distance $\xi$. Thus from this geometry  using small angle and weak field approximation, we can define the ray tracing equation. And the deflection angle is given by this formula for a point like lens. In the case of Sun, the deflection angle is comes out equal to 1.7", which was well verified by Prof Eddington and their group in 1919. This was the first observational test of general theory of relativity.

\item[\ding{74} Slide-10]
Depending on the alignment of lens, observer and source, we can get different type of images from lensing effect. For example, if there is a perfect alignment of lens, observer and source, and spherical symmatry of lens, then a ring like images is formed as shown in the Left figure. This ring like image is called Einstein ring with radius $\theta_E$ which is given by this formula. On the other hand, here in this figure, we can see that if there is no perfect alignment then a arc like images are formed. Finally, using Einstein radius, we can find the solution of lens or ray tracing equation.

\item[\ding{74} Slide-11]
 In strong gravitational lensing, there are two important quantities, which helps to study the lens galaxy profile and also help us to constrain the cosmological parameters. Therefore, our talk is divided into two parts. In the first part, we will discuss distance ratio. And time delay distance will discuss in the second part of talk. 

\item[\ding{74} Slide-12]
\hspace{5cm} Part-A Distance ratio.

 \item[\ding{74} Slide-13]
So far, we discussed lensing effect for a point mass objects. Obviously, a point mass lens is not a good model for a galaxy lens. Now In the first model of the lens galaxy we assume that all the stars and other mass components are like a particle of a ideal gas  with density $\rho$ and their temperature is related to the velocity-dispersion $\sigma_v$. This kind of mass distribution is known as the Singular Isothermal Sphere. Using the Hydrostatic Equilibrium condition, we calculate the density of SIS model and deflection angle as written over here. This density tends to $\infty$ as $r\rightarrow 0$. Therefore, it is called singular isothermal sphere. Using the Einstein Radius, one can define the ratio of angular diameters.

 \item[\ding{74} Slide-14]
So in the case of SIS lens profile we saw that variation of density with radial coordinate is $1/r^2$. On the other hand, In the second model of the lens galaxy we assume that mass density varies as $1/r^\gamma$ rather than $1/r^2$. After considering all these point and using Spherical Symmetric Jeans Equations, the distance ratio can be found. where lens profile parameters dependent function is defined. 

 \item[\ding{74} Slide-15]
Next, we consider a more general and complex model of a lens galaxy, which allows the luminosity density profile different from the total density profile, which is due to both luminous and dark matter. We also consider the anisotropic distribution of 3D velocity dispersion. This shows that the radial and tangential velocity dispersions might be different. After considering all these point and using Spherical Symmetric Jeans Equations, the distance ratio can be found. where lens profile parameters dependent function is defined.

 \item[\ding{74} Slide-16]
So far, we have defined the distance ratio for a general lens profile. One can reduce this lens profile into a power law and SIS under some assumptions. For power law lens profile, only one lens parameter will take part in a distance ratio and for SIS $\gamma =\delta =2$ and $\beta=0$. So finally distance ratio formula reduces to the one we have already seen above.


 \end{itemize}
Now we have to define the observable distance ratio denoted $\boldsymbol{d}_{\mathbf{A} \mathbf{R}}$, which includes the lens parameters, $\alpha,~\delta$ and $\beta$.

\begin{itemize}
 \item[\ding{74} Slide-17]
As we know in a Euclidean geometry the distances are additive. But this same distance relation is modified in the case of non-flat universe  by taking care of cosmic curvature parameter. And this relation can be rewritten with the help of a well established relation i.e. Cosmic Distance Duality Relation. Thus finally we can define the ratio of distance as shown here, which includes the cosmic curvature parameter and luminosity distances at lens and source redshift.

 \item[\ding{74} Slide-18]
In the theoretical construction of distance ratio, we have to estimate the luminosity distance at a given value of lens and source redshift. For this, we use the Pantheon and GRBs dataset upto redshift of value 3.6 because in this analysis source redshift is upto 3.595. For redshift matching of Pantheon and GRBs dataset to lens and source redshift, we fit a second order polynomial which is shown in this figure. 1 and 2 $\sigma$ regions are also shown. We also plotted the luminosity distance for a standard $\Lambda$CDM model by considering $H_0=74.02\pm 1.42$ km/sec/Mpc. 

\item[\ding{74} Slide-19]
For the distance ratio, we use the largest dataset with 161 datapoints of SGL systems. In this dataset lenses are upto redshift 1 and sources are upto redshift 3.6. \\
So far, we constructed the distance ratio observationally from SGL dataset and theoretically from Distance sum rule method. These distances ratios have unknown quantities that are cosmic curvature parameter and lens profile parameters. Therefore, we estimate these unknown parameters by maximising the likelihood. 
\end{itemize}
Next we'll discuss the results obtained from this analysis for SIS, PSL and EPL models.
\begin{itemize}
\item[\ding{74} Slide-20]
We have two unknown parameters for the SIS lens model that is the cosmic curvature parameter and a lens parameter that is $f_e$. In our analysis we find a flat universe is not accommodated at $68\%$CL. On the other hand we find $f_e$ nearly equally to one that is consistent with SIS model of a lens galaxy. 

\item[\ding{74} Slide-21]
In the second lens profile that is power law spherical lens: we have cosmic curvature parameter and a lens parameter. The power index of total density of mass distribution. Ideally, it should be \textbf{two.} Here, we choose a redshift evolving power index $\gamma=\gamma_0+\gamma_1z$. We find a closed universe but also flat universe is accommodated within $68\%$ CL. On other hand, a non-zero value of $\gamma_1$ indicates that the mass density profile of massive galaxies has become steeper over cosmic time.

\item[\ding{74} {\footnotesize Slide-22-24}]
Finally, we use a EPL lens profile. In which we assume two types of density parameters with their power index $\gamma$ that includes both luminous and dark matter and $\delta$ which tells about only luminous part of matter. Here we use three parametrisation of $\gamma$. In first parametrisation we assume an arbitrary constant $\gamma$ and in P2 and P3 $\gamma$ is evolving with redshift in a different ways. In all three parametersation, we find a closed universe is preferred and also non-zero a non-zero value of $\gamma_1$ indicating the lens galaxy is evolving with redshift. And different value of $\gamma$ and $\delta$ indicate a mass other than luminous content present in the universe.

\item[\ding{74} Slide-25] 
Will read the conclusions.

\item[\ding{74} Slide-26]
\hspace{5cm}  Part-B Time Delay Distance.

\item[\ding{74} Slide-27]
In gravitational lensing, we can imagine that their is a time difference between the unperturbed lights and deflected ones. This time difference or time delay has two components; that is (as shown here). The 1st one is geometrical time delay and this is becuase there is different path length of deflected light rays compared to the unperturbed ones. And we can see here that this is proportional to squared angular separation between the intrinsic position of source and its image location. 

And the second one that is Gravi. time delay comes from the slowing down of the photons traveling through the gravitational field of lens and is therefore related to the lensing potential as shown here.

%When a background object (source) emit a light rays, will take different paths through space- time at the different image positions. Because these paths have different path length and also passes through a different gravitational potentials. Time delay due to different path length is called geometric time delay and gravitational or Shapiro time delay is because of different gravitational potential of lens. So, light rays which were emitted at the same times, will reach to earth or observer at different times. Therefore, there is delay in time between multiple images. If the source is variable, this time-delay can be measured by monitoring the lens which gives us the information about the flux variations corresponding to the same source event. This time-delay is related to a quantity which is referred as ``time-delay distance" is used to estimate the cosmological parameters.

\item[\ding{74} Slide-28]
In the figure, it is shown that there are two images $I_1$ and $I_2$, which are located at different position with different path length from observer. For a simplest case of lens model in which the mass density profile of a
lens galaxy is given by an SIS, time delay between two images is given by this formula, where, $\theta_i$ and  $\theta_j$ are angular positions of two images and $z_l$ the redshift of lens galaxy.

\item[\ding{74} Slide-29]
As we already discussed distance sum rule method of the distance ratio in SGL systems. In the same fashion, we can rewrite DSR method in term of time delay distance. Where we also consider a distance duality parameter $\eta$, which will help us to check this cosmic distance duality relation.

\item[\ding{74} Slide-30]
For the time delay distance part, we use the dataset with 12 datapoints of SGL systems. In this dataset lenses are upto redshift 0.890 and sources are upto redshift 2.719. Along with this dataset, we also considered the latest dataset of H0LiCOW with 6 lens. So far, we constructed the time delay distance observationally from SGL dataset and theoretically from Distance sum rule method. These time delay distances have unknown quantities that are cosmic curvature parameter and cosmic distance duality  parameters. Therefore, we estimate these unknown parameters by maximising the likelihood or minimising the $\chi^2$.

\item[\ding{74} {\footnotesize Slide-31-33}]
In this work,  here we use three parametrisation of $\eta$. In first parametrisation we assume an arbitrary constant $\eta$ and in P2 and P3 $\eta$ is evolving with redshift in a different ways. In all three parametrisation,  the best fit value of $\Omega_{k0}$ suggest an open universe, but also a flat universe is also accommodated  in $68\%$ CL.

\item[\ding{74} Slide-34] 
Will read the conclusions.

\item[\ding{74} Slide-35] 
Here are some references that we used for this work.

 \item[\ding{74} {{\normalsize Slide-0b}}]
Here are some list of attended workshop and international conferences.
\begin{center}
Thank you for your attention.
\end{center}
\end{itemize}




\end{document}
